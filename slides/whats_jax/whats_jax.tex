\documentclass[
    xcolor={svgnames,dvipsnames},
    hyperref={colorlinks, citecolor=DeepPink4, linkcolor=DarkRed, urlcolor=DarkBlue}
    ]{beamer}  % for hardcopy add 'trans'

\mode<presentation>
{
  \usetheme{Singapore}
  % or ...
  \setbeamercovered{transparent}
  % or whatever (possibly just delete it)
}



\addtobeamertemplate{navigation symbols}{}{%
    \usebeamerfont{footline}%
    \usebeamercolor[fg]{footline}%
    \hspace{1em}%
    \insertframenumber/\inserttotalframenumber
}



\usepackage{fontspec} 
%\usepackage[xcharter]{newtxmath}
%\setmainfont{XCharter}
\usepackage{unicode-math}
%\setmathfont{XCharter-Math.otf}
\setmonofont{DejaVu Sans Mono}[Scale=MatchLowercase] % provides unicode characters 

\usepackage{tikz}
\usetikzlibrary{matrix, shapes, arrows.meta, positioning, fit, backgrounds, calc}

% \usetikzlibrary{matrix, arrows.meta, positioning, calc, backgrounds}

% for tikz
\usepackage{pgfplots}
\usepgfplotslibrary{fillbetween}
\pgfplotsset{compat=1.16}

\usepackage{varwidth}
\usepackage{minted}
\usemintedstyle{friendly}
\setminted[python]{
  fontsize=\small,
  baselinestretch=1.2,
  bgcolor=codebg,
  linenos=false,
  breaklines=true,
  frame=none
}
\setminted[matlab]{
  fontsize=\small,
  baselinestretch=1.2,
  bgcolor=codebg,
  linenos=false,
  breaklines=true,
  frame=none
}
\setminted[julia]{
  fontsize=\small,
  baselinestretch=1.2,
  bgcolor=codebg,
  linenos=false,
  breaklines=true,
  frame=none
}
%\setminted{mathescape, frame=lines, framesep=3mm}
%\newminted{python}{}
%\newminted{c}{mathescape,frame=lines,framesep=4mm,bgcolor=bg}
%\newminted{java}{mathescape,frame=lines,framesep=4mm,bgcolor=bg}
%\newminted{julia}{mathescape,frame=lines,framesep=4mm,bgcolor=bg}
%\newminted{ipython}{mathescape,frame=lines,framesep=4mm,bgcolor=bg}

\usepackage{graphicx}
\usepackage{amsmath, amssymb, amsthm}
\usepackage{bbm}
\usepackage{mathrsfs}
\usepackage{xcolor}
\usepackage{fancyvrb}


% Quotes at start of chapters / sections
\usepackage{epigraph}  
\renewcommand{\epigraphwidth}{6in}

%% Fonts

%\usepackage[T1]{fontenc}
\usepackage{mathpazo}
%\usepackage{fontspec}
%\defaultfontfeatures{Ligatures=TeX}
%\setsansfont[Scale=MatchLowercase]{DejaVu Sans}
%\setmonofont[Scale=MatchLowercase]{DejaVu Sans Mono}
%\setmathfont{Asana Math}
%\setmainfont{Optima}
%\setmathrm{Optima}
%\setboldmathrm[BoldFont={Optima ExtraBlack}]{Optima Bold}

% Some colors

\definecolor{containerblue}{RGB}{66, 133, 244}
\definecolor{leafgreen}{RGB}{52, 168, 83}
\definecolor{textgray}{RGB}{51, 51, 51}
\definecolor{backgroundgray}{RGB}{248, 249, 250}
\definecolor{codebg}{RGB}{241, 241, 241}
\definecolor{aquamarine}{RGB}{69,139,116}
\definecolor{midnightblue}{RGB}{25,25,112}
\definecolor{darkslategrey}{RGB}{47,79,79}
\definecolor{darkorange4}{RGB}{139,90,0}
\definecolor{dogerblue}{RGB}{24,116,205}
\definecolor{blue2}{RGB}{0,0,238}
\definecolor{bg}{rgb}{0.95,0.95,0.95}
\definecolor{DarkOrange1}{RGB}{255,127,0}
\definecolor{ForestGreen}{RGB}{34,139,34}
\definecolor{DarkRed}{RGB}{139, 0, 0}
\definecolor{DarkBlue}{RGB}{0, 0, 139}
\definecolor{Blue}{RGB}{0, 0, 255}
\definecolor{Brown}{RGB}{165,42,42}


\setlength{\parskip}{1.5ex plus0.5ex minus0.5ex}

%\renewcommand{\baselinestretch}{1.05}
%\setlength{\parskip}{1.5ex plus0.5ex minus0.5ex}
%\setlength{\parindent}{0pt}

% Typesetting code
\definecolor{bg}{rgb}{0.95,0.95,0.95}
\newcommand{\Fact}{\textcolor{Brown}{\bf Fact. }}
\newcommand{\Facts}{\textcolor{Brown}{\bf Facts }}
\newcommand{\keya}{\textcolor{turquois4}{\bf Key Idea. }}
\newcommand{\Factnodot}{\textcolor{Brown}{\bf Fact }}
\newcommand{\Eg}{\textcolor{ForestGreen}{Example. }}
\newcommand{\Egs}{\textcolor{ForestGreen}{Examples. }}
\newcommand{\Ex}{{\bf Ex. }}



\renewcommand{\theFancyVerbLine}{\sffamily
    \textcolor[rgb]{0.5,0.5,1.0}{\scriptsize {\arabic{FancyVerbLine}}}}

\newcommand{\navy}[1]{\textcolor{DarkBlue}{\bf #1}}
\newcommand{\brown}[1]{\textcolor{Brown}{\sf #1}}
\newcommand{\green}[1]{\textcolor{ForestGreen}{\sf #1}}
\newcommand{\blue}[1]{\textcolor{Blue}{\sf #1}}
\newcommand{\emp}[1]{\textcolor{DarkOrange1}{\bf #1}}
\newcommand{\red}[1]{\textcolor{Red}{\bf #1}}

% Symbols, redefines, etc.

\newcommand{\code}[1]{\texttt{#1}}

\newcommand{\argmax}{\operatornamewithlimits{argmax}}
\newcommand{\argmin}{\operatornamewithlimits{argmin}}

\DeclareMathOperator{\cl}{cl}
\DeclareMathOperator{\interior}{int}
\DeclareMathOperator{\Prob}{Prob}
\DeclareMathOperator{\determinant}{det}
\DeclareMathOperator{\trace}{trace}
\DeclareMathOperator{\Span}{span}
\DeclareMathOperator{\rank}{rank}
\DeclareMathOperator{\cov}{cov}
\DeclareMathOperator{\corr}{corr}
\DeclareMathOperator{\var}{var}
\DeclareMathOperator{\mse}{mse}
\DeclareMathOperator{\se}{se}
\DeclareMathOperator{\row}{row}
\DeclareMathOperator{\col}{col}
\DeclareMathOperator{\range}{rng}
\DeclareMathOperator{\dimension}{dim}
\DeclareMathOperator{\bias}{bias}


% mics short cuts and symbols
\newcommand{\st}{\ensuremath{\ \mathrm{s.t.}\ }}
\newcommand{\setntn}[2]{ \{ #1 : #2 \} }
\newcommand{\cf}[1]{ \lstinline|#1| }
\newcommand{\fore}{\therefore \quad}
\newcommand{\tod}{\stackrel { d } {\to} }
\newcommand{\toprob}{\stackrel { p } {\to} }
\newcommand{\toms}{\stackrel { ms } {\to} }
\newcommand{\eqdist}{\stackrel {\textrm{ \scriptsize{d} }} {=} }
\newcommand{\iidsim}{\stackrel {\textrm{ {\sc iid }}} {\sim} }
\newcommand{\1}{\mathbbm 1}
\newcommand{\dee}{\,{\rm d}}
\newcommand{\given}{\, | \,}
\newcommand{\la}{\langle}
\newcommand{\ra}{\rangle}

\newcommand{\boldA}{\mathbf A}
\newcommand{\boldB}{\mathbf B}
\newcommand{\boldC}{\mathbf C}
\newcommand{\boldD}{\mathbf D}
\newcommand{\boldM}{\mathbf M}
\newcommand{\boldP}{\mathbf P}
\newcommand{\boldQ}{\mathbf Q}
\newcommand{\boldI}{\mathbf I}
\newcommand{\boldX}{\mathbf X}
\newcommand{\boldY}{\mathbf Y}
\newcommand{\boldZ}{\mathbf Z}

\newcommand{\bSigmaX}{ {\boldsymbol \Sigma_{\hboldbeta}} }
\newcommand{\hbSigmaX}{ \mathbf{\hat \Sigma_{\hboldbeta}} }

\newcommand{\RR}{\mathbbm R}
\newcommand{\NN}{\mathbbm N}
\newcommand{\PP}{\mathbbm P}
\newcommand{\EE}{\mathbbm E \,}
\newcommand{\XX}{\mathbbm X}
\newcommand{\ZZ}{\mathbbm Z}
\newcommand{\QQ}{\mathbbm Q}

\newcommand{\fF}{\mathcal F}
\newcommand{\dD}{\mathcal D}
\newcommand{\lL}{\mathcal L}
\newcommand{\gG}{\mathcal G}
\newcommand{\hH}{\mathcal H}
\newcommand{\nN}{\mathcal N}
\newcommand{\pP}{\mathcal P}




\title{What's JAX}



\author{John Stachurski}


\date{2025}


\begin{document}

\begin{frame}
  \titlepage
\end{frame}



\begin{frame}
    \frametitle{Topics}

    \begin{itemize}
        \item Foo
        \vspace{0.5em}
        \item Bar
        \vspace{0.5em}
    \end{itemize}

\end{frame}



    

\begin{frame}
    \frametitle{Focus on JAX}

    \url{https://jax.readthedocs.io/en/latest/}
    
            \vspace{0.5em}
    \begin{itemize}
        \item \brown{J}ust-in-time compilation
            \vspace{0.5em}
        \item \brown{A}utomatic differentiation
            \vspace{0.5em}
        \item \brown{X}ccelerated linear algebra
    \end{itemize}

            \vspace{0.5em}
            \vspace{0.5em}

\end{frame}



\begin{frame}[fragile]
    
    \vspace{-1em}
    \begin{minted}{python}
import jax.numpy as jnp
from jax import grad, jit

def f(θ, x):
  for W, b in θ:
    w = W @ x + b
    x = jnp.tanh(w)  
  return x

def loss(θ, x, y):
  return jnp.sum((y - f(θ, x))**2)

grad_loss = jit(grad(loss))  # Now use gradient descent 
    \end{minted}

\end{frame}


\begin{frame}
    
    \Eg AlphaFold3 (Google JAX)

        \vspace{0.5em}
    \textbf{Highly accurate protein structure prediction with AlphaFold}

        \vspace{0.5em}
    John Jumper, Richard Evans, Alexander Pritzel, Tim Green, Michael Figurnov,
    Olaf Ronneberger, Kathryn Tunyasuvunakool,\ldots 

        \vspace{0.5em}
    \underline{Nature} Vol.\ 596 (2021)

    \vspace{0.5em}
    \vspace{0.5em}
    \vspace{0.5em}
    \vspace{0.5em}
    \begin{itemize}
        \item Citation count $= ~30$K
        \vspace{0.5em}
        \item Nobel Prize in Chemistry 2024
    \end{itemize}

\end{frame}


\begin{frame}
    \frametitle{Functional Programming}
    
    JAX adopts a \emp{functional programming style}

    \vspace{0.5em}
    \vspace{0.5em}
    \vspace{0.5em}
    \vspace{0.5em}
    Key feature: Functions are pure

    \begin{itemize}
        \item Deterministic: same input $\implies$ same output 
        \item Have no side effects (don't modify state outside their scope)
    \end{itemize}

\end{frame}


\begin{frame}[fragile]

    A non-pure function

    \begin{minted}{python}
tax_rate = 0.1  # Global 
price = 10.0    # Global

def add_tax_non_pure():
    global price                
    # The next line both accesses and modifies global state
    price = price * (1 + tax_rate)    
    return price 
    \end{minted}
    
\end{frame}


\begin{frame}[fragile]

    A pure function

    \begin{minted}{python}

def add_tax_non_pure(price, tax_rate=0.1):
    price = price * (1 + tax_rate)    
    return price 
    \end{minted}
    
\end{frame}


\begin{frame}

    General advantages:

    \begin{itemize}
        \item Helps testing: each function can operate in isolation
        \item Data dependencies are explicit, which helps with understanding and optimizing complex computations 
        \item Promotes deterministic behavior and hence reproducibility
        \item Prevents subtle bugs that arise from mutating shared state
    \end{itemize}

\end{frame}



\begin{frame}
    
    Advantages for JAX:

     \begin{itemize}
         \item Functional programming facilitates autodiff because
             pure functions are more straightforward to differentiate (don't mod
             external state
         \item Pure functions are easier to
             parallelize and optimize for hardware accelerators like GPUs (don't
             depend on shared mutable state, more independence)
         \item Transformations can be composed cleanly with multiple
             transformations yielding predictable results
        \item Portability across hardware: The functional approach helps JAX
            create code that can run efficiently across different hardware
            accelerators without requiring hardware-specific implementations.
     \end{itemize}


\end{frame}

\begin{frame}
    \frametitle{JAX PyTrees}

    A PyTree is a concept in the JAX library that refers to a tree-like data structure built from Python containers.

    \Egs

    \begin{itemize}
        \item A dictionary of lists of parameters
        \item A list of dictionaries of parameters, etc.
    \end{itemize}

    JAX can

    \begin{itemize}
        \item apply functions to all leaves in a PyTree structure
        \item differentiate functions with respect to the leaves of PyTrees
        \item etc.
    \end{itemize}

\end{frame}


\begin{frame}
    
    
    \resizebox{1.0\textwidth}{!}{
        % JAX PyTree TikZ Diagram with Pygments Syntax Highlighting
% To use this in your LaTeX document:
% 1. Ensure you have the required packages in your preamble: tikz, minted, varwidth
% 2. Load TikZ libraries: shapes, arrows, positioning, fit, backgrounds, calc
% 3. Define the colors and minted settings in your preamble
% 4. Place this code inside a figure environment

% Required packages for your preamble:
% \usepackage{tikz}
% \usepackage{minted}
% \usepackage{varwidth}
% \usetikzlibrary{shapes,arrows,positioning,fit,backgrounds,calc}

% Color definitions for your preamble:
% \definecolor{containerblue}{RGB}{66, 133, 244}
% \definecolor{leafgreen}{RGB}{52, 168, 83}
% \definecolor{textgray}{RGB}{51, 51, 51}
% \definecolor{backgroundgray}{RGB}{248, 249, 250}
% \definecolor{codebg}{RGB}{241, 241, 241}

% Minted settings for your preamble:
% \setminted[python]{
%   fontsize=\small,
%   baselinestretch=1.2,
%   bgcolor=codebg,
%   linenos=false,
%   breaklines=true,
%   frame=none
% }

% Place the following code in your document inside a figure environment:
% \begin{figure}
%   \centering
%   ... code below ...
%   \caption{JAX PyTree Structure Visualization}
%   \label{fig:jax-pytree}
% \end{figure}

\begin{tikzpicture}[
    % Node styles
    container/.style={draw=containerblue!70!black, fill=containerblue, rounded corners=3pt, minimum width=3cm, minimum height=1cm, text=white, font=\sffamily\bfseries},
    leaf/.style={draw=leafgreen!70!black, fill=leafgreen, rounded corners=3pt, minimum width=2cm, minimum height=1cm, text=white, font=\sffamily\bfseries},
    label/.style={font=\sffamily\small, text=textgray},
    level distance=2cm,
    sibling distance=3cm
]

% Background
\node[rectangle, fill=backgroundgray, minimum width=20cm, minimum height=15cm] (background) at (0,0) {};

% Title
\node[font=\sffamily\bfseries\Large, text=textgray] at (0,6.5) {JAX PyTree Structure};

% Sample code with syntax highlighting
\node[anchor=north] (code) at (0,5.5) {
  \begin{varwidth}{14cm}
    \begin{minted}{python}
pytree = {
    "a": [1, 2, 3],
    "b": {"c": jnp.array([4, 5]), "d": jnp.array([[6, 7], [8, 9]])}
}
    \end{minted}
  \end{varwidth}
};

% Add a border around the code
\begin{scope}[on background layer]
\draw[rounded corners=3pt, draw=black!20] 
  ([shift={(-0.5em,-0.5em)}]code.south west) rectangle 
  ([shift={(0.5em,0.5em)}]code.north east);
\end{scope}

\path (code.south) -- ++(0,-1.5) coordinate (after_code);

\node[container] (root) at (0,2) {dict};

% First level children
\node[container] (list) at (-4,1) {list};
\node[label] at (-4,1.8) {"a"};
\node[container] (dict) at (4,1) {dict};
\node[label] at (4,1.8) {"b"};

% Second level - list elements
\node[leaf] (n1) at (-6,-1) {1};
\node[leaf] (n2) at (-4,-1) {2};
\node[leaf] (n3) at (-2,-1) {3};

% Second level - dict elements
\node[container] (array1) at (2,-1) {array};
\node[label] at (2,-0.2) {"c"};
\node[container] (array2) at (6,-1) {array};
\node[label] at (6,-0.2) {"d"};

% Array values (leaf nodes)
\node[leaf, text width=2.5cm, align=center] (val1) at (2,-3) {[4, 5]};
\node[leaf, text width=3cm, align=center] (val2) at (6,-3) {[[6, 7], [8, 9]]};

% Edges connecting nodes
\draw[->, >=stealth, thick, textgray] (root) -- (list);
\draw[->, >=stealth, thick, textgray] (root) -- (dict);
\draw[->, >=stealth, thick, textgray] (list) -- (n1);
\draw[->, >=stealth, thick, textgray] (list) -- (n2);
\draw[->, >=stealth, thick, textgray] (list) -- (n3);
\draw[->, >=stealth, thick, textgray] (dict) -- (array1);
\draw[->, >=stealth, thick, textgray] (dict) -- (array2);
\draw[->, >=stealth, thick, textgray] (array1) -- (val1);
\draw[->, >=stealth, thick, textgray] (array2) -- (val2);

% Legend
\node[container, minimum width=1cm, minimum height=0.6cm] (legend1) at (-7,-5) {};
\node[text=textgray, font=\sffamily, anchor=west] at (-6.5,-5) {Container nodes (dict, list, tuple)};

\node[leaf, minimum width=1cm, minimum height=0.6cm] (legend2) at (-7,-6) {};
\node[text=textgray, font=\sffamily, anchor=west] at (-6.5,-6) {Leaf nodes (arrays, scalars)};

% Additional explanation
\node[text=textgray, font=\sffamily\bfseries, align=center] at (0,-4.5) {JAX PyTrees: Nested containers with leaves as arrays or scalars};

\end{tikzpicture}

    }

\end{frame}


\begin{frame}[fragile]
    
    \begin{minted}{python}
# Apply gradient updates to all parameters
def sgd_update(params, grads, learning_rate):
    return jax.tree_map(
        lambda p, g: p - learning_rate * g, 
        params, 
        grads
    )

# Calculate gradients (PyTree with same structure as params)
grads = jax.grad(loss_fn)(params, inputs, targets)

# Update all parameters at once
updated_params = sgd_update(params, grads, 0.01)    
    \end{minted}

\end{frame}

