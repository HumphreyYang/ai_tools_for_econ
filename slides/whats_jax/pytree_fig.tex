% JAX PyTree TikZ Diagram with Pygments Syntax Highlighting
% To use this in your LaTeX document:
% 1. Ensure you have the required packages in your preamble: tikz, minted, varwidth
% 2. Load TikZ libraries: shapes, arrows, positioning, fit, backgrounds, calc
% 3. Define the colors and minted settings in your preamble
% 4. Place this code inside a figure environment

% Required packages for your preamble:
% \usepackage{tikz}
% \usepackage{minted}
% \usepackage{varwidth}
% \usetikzlibrary{shapes,arrows,positioning,fit,backgrounds,calc}

% Color definitions for your preamble:
% \definecolor{containerblue}{RGB}{66, 133, 244}
% \definecolor{leafgreen}{RGB}{52, 168, 83}
% \definecolor{textgray}{RGB}{51, 51, 51}
% \definecolor{backgroundgray}{RGB}{248, 249, 250}
% \definecolor{codebg}{RGB}{241, 241, 241}

% Minted settings for your preamble:
% \setminted[python]{
%   fontsize=\small,
%   baselinestretch=1.2,
%   bgcolor=codebg,
%   linenos=false,
%   breaklines=true,
%   frame=none
% }

% Place the following code in your document inside a figure environment:
% \begin{figure}
%   \centering
%   ... code below ...
%   \caption{JAX PyTree Structure Visualization}
%   \label{fig:jax-pytree}
% \end{figure}

\begin{tikzpicture}[
    % Node styles
    container/.style={draw=containerblue!70!black, fill=containerblue, rounded corners=3pt, minimum width=3cm, minimum height=1cm, text=white, font=\sffamily\bfseries},
    leaf/.style={draw=leafgreen!70!black, fill=leafgreen, rounded corners=3pt, minimum width=2cm, minimum height=1cm, text=white, font=\sffamily\bfseries},
    label/.style={font=\sffamily\small, text=textgray},
    level distance=2cm,
    sibling distance=3cm
]

% Background
\node[rectangle, fill=backgroundgray, minimum width=20cm, minimum height=15cm] (background) at (0,0) {};

% Title
\node[font=\sffamily\bfseries\Large, text=textgray] at (0,6.5) {JAX PyTree Structure};

% Sample code with syntax highlighting
\node[anchor=north] (code) at (0,5.5) {
  \begin{varwidth}{14cm}
    \begin{minted}{python}
pytree = {
    "a": [1, 2, 3],
    "b": {"c": jnp.array([4, 5]), "d": jnp.array([[6, 7], [8, 9]])}
}
    \end{minted}
  \end{varwidth}
};

% Add a border around the code
\begin{scope}[on background layer]
\draw[rounded corners=3pt, draw=black!20] 
  ([shift={(-0.5em,-0.5em)}]code.south west) rectangle 
  ([shift={(0.5em,0.5em)}]code.north east);
\end{scope}

\path (code.south) -- ++(0,-1.5) coordinate (after_code);

\node[container] (root) at (0,2) {dict};

% First level children
\node[container] (list) at (-4,1) {list};
\node[label] at (-4,1.8) {"a"};
\node[container] (dict) at (4,1) {dict};
\node[label] at (4,1.8) {"b"};

% Second level - list elements
\node[leaf] (n1) at (-6,-1) {1};
\node[leaf] (n2) at (-4,-1) {2};
\node[leaf] (n3) at (-2,-1) {3};

% Second level - dict elements
\node[container] (array1) at (2,-1) {array};
\node[label] at (2,-0.2) {"c"};
\node[container] (array2) at (6,-1) {array};
\node[label] at (6,-0.2) {"d"};

% Array values (leaf nodes)
\node[leaf, text width=2.5cm, align=center] (val1) at (2,-3) {[4, 5]};
\node[leaf, text width=3cm, align=center] (val2) at (6,-3) {[[6, 7], [8, 9]]};

% Edges connecting nodes
\draw[->, >=stealth, thick, textgray] (root) -- (list);
\draw[->, >=stealth, thick, textgray] (root) -- (dict);
\draw[->, >=stealth, thick, textgray] (list) -- (n1);
\draw[->, >=stealth, thick, textgray] (list) -- (n2);
\draw[->, >=stealth, thick, textgray] (list) -- (n3);
\draw[->, >=stealth, thick, textgray] (dict) -- (array1);
\draw[->, >=stealth, thick, textgray] (dict) -- (array2);
\draw[->, >=stealth, thick, textgray] (array1) -- (val1);
\draw[->, >=stealth, thick, textgray] (array2) -- (val2);

% Legend
\node[container, minimum width=1cm, minimum height=0.6cm] (legend1) at (-7,-5) {};
\node[text=textgray, font=\sffamily, anchor=west] at (-6.5,-5) {Container nodes (dict, list, tuple)};

\node[leaf, minimum width=1cm, minimum height=0.6cm] (legend2) at (-7,-6) {};
\node[text=textgray, font=\sffamily, anchor=west] at (-6.5,-6) {Leaf nodes (arrays, scalars)};

% Additional explanation
\node[text=textgray, font=\sffamily\bfseries, align=center] at (0,-4.5) {JAX PyTrees: Nested containers with leaves as arrays or scalars};

\end{tikzpicture}
