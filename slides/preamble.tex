\documentclass[
    xcolor={svgnames,dvipsnames},
    hyperref={colorlinks, citecolor=DeepPink4, linkcolor=DarkRed, urlcolor=DarkBlue}
    ]{beamer}  % for hardcopy add 'trans'

\mode<presentation>
{
  \usetheme{Singapore}
  % or ...
  \setbeamercovered{transparent}
  % or whatever (possibly just delete it)
}



\addtobeamertemplate{navigation symbols}{}{%
    \usebeamerfont{footline}%
    \usebeamercolor[fg]{footline}%
    \hspace{1em}%
    \insertframenumber/\inserttotalframenumber
}



\usepackage{fontspec} 
%\usepackage[xcharter]{newtxmath}
%\setmainfont{XCharter}
\usepackage{unicode-math}
%\setmathfont{XCharter-Math.otf}
\setmonofont{DejaVu Sans Mono}[Scale=MatchLowercase] % provides unicode characters 

\usepackage{tikz}
\usepackage{pgfplots}
\usepgfplotslibrary{fillbetween}
\pgfplotsset{compat=1.16}
\usepackage{varwidth}
\usetikzlibrary{shapes,arrows,positioning,fit,backgrounds,calc}

\usepackage{minted}
\usemintedstyle{friendly}
\setminted[python]{
  fontsize=\small,
  baselinestretch=1.2,
  bgcolor=codebg,
  linenos=false,
  breaklines=true,
  frame=none
}
%\setminted{mathescape, frame=lines, framesep=3mm}
%\newminted{python}{}
%\newminted{c}{mathescape,frame=lines,framesep=4mm,bgcolor=bg}
%\newminted{java}{mathescape,frame=lines,framesep=4mm,bgcolor=bg}
%\newminted{julia}{mathescape,frame=lines,framesep=4mm,bgcolor=bg}
%\newminted{ipython}{mathescape,frame=lines,framesep=4mm,bgcolor=bg}

\usepackage{graphicx}
\usepackage{amsmath, amssymb, amsthm}
\usepackage{bbm}
\usepackage{mathrsfs}
\usepackage{xcolor}
\usepackage{fancyvrb}


% Quotes at start of chapters / sections
\usepackage{epigraph}  
\renewcommand{\epigraphwidth}{6in}

%% Fonts

%\usepackage[T1]{fontenc}
\usepackage{mathpazo}
%\usepackage{fontspec}
%\defaultfontfeatures{Ligatures=TeX}
%\setsansfont[Scale=MatchLowercase]{DejaVu Sans}
%\setmonofont[Scale=MatchLowercase]{DejaVu Sans Mono}
%\setmathfont{Asana Math}
%\setmainfont{Optima}
%\setmathrm{Optima}
%\setboldmathrm[BoldFont={Optima ExtraBlack}]{Optima Bold}

% Some colors

\definecolor{containerblue}{RGB}{66, 133, 244}
\definecolor{leafgreen}{RGB}{52, 168, 83}
\definecolor{textgray}{RGB}{51, 51, 51}
\definecolor{backgroundgray}{RGB}{248, 249, 250}
\definecolor{codebg}{RGB}{241, 241, 241}
\definecolor{aquamarine}{RGB}{69,139,116}
\definecolor{midnightblue}{RGB}{25,25,112}
\definecolor{darkslategrey}{RGB}{47,79,79}
\definecolor{darkorange4}{RGB}{139,90,0}
\definecolor{dogerblue}{RGB}{24,116,205}
\definecolor{blue2}{RGB}{0,0,238}
\definecolor{bg}{rgb}{0.95,0.95,0.95}
\definecolor{DarkOrange1}{RGB}{255,127,0}
\definecolor{ForestGreen}{RGB}{34,139,34}
\definecolor{DarkRed}{RGB}{139, 0, 0}
\definecolor{DarkBlue}{RGB}{0, 0, 139}
\definecolor{Blue}{RGB}{0, 0, 255}
\definecolor{Brown}{RGB}{165,42,42}


\setlength{\parskip}{1.5ex plus0.5ex minus0.5ex}

%\renewcommand{\baselinestretch}{1.05}
%\setlength{\parskip}{1.5ex plus0.5ex minus0.5ex}
%\setlength{\parindent}{0pt}

% Typesetting code
\definecolor{bg}{rgb}{0.95,0.95,0.95}
\newcommand{\Fact}{\textcolor{Brown}{\bf Fact. }}
\newcommand{\Facts}{\textcolor{Brown}{\bf Facts }}
\newcommand{\keya}{\textcolor{turquois4}{\bf Key Idea. }}
\newcommand{\Factnodot}{\textcolor{Brown}{\bf Fact }}
\newcommand{\Eg}{\textcolor{ForestGreen}{Example. }}
\newcommand{\Egs}{\textcolor{ForestGreen}{Examples. }}
\newcommand{\Ex}{{\bf Ex. }}



\renewcommand{\theFancyVerbLine}{\sffamily
    \textcolor[rgb]{0.5,0.5,1.0}{\scriptsize {\arabic{FancyVerbLine}}}}

\newcommand{\navy}[1]{\textcolor{DarkBlue}{\bf #1}}
\newcommand{\brown}[1]{\textcolor{Brown}{\sf #1}}
\newcommand{\green}[1]{\textcolor{ForestGreen}{\sf #1}}
\newcommand{\blue}[1]{\textcolor{Blue}{\sf #1}}
\newcommand{\emp}[1]{\textcolor{DarkOrange1}{\bf #1}}
\newcommand{\red}[1]{\textcolor{Red}{\bf #1}}

% Symbols, redefines, etc.

\newcommand{\code}[1]{\texttt{#1}}

\newcommand{\argmax}{\operatornamewithlimits{argmax}}
\newcommand{\argmin}{\operatornamewithlimits{argmin}}

\DeclareMathOperator{\cl}{cl}
\DeclareMathOperator{\interior}{int}
\DeclareMathOperator{\Prob}{Prob}
\DeclareMathOperator{\determinant}{det}
\DeclareMathOperator{\trace}{trace}
\DeclareMathOperator{\Span}{span}
\DeclareMathOperator{\rank}{rank}
\DeclareMathOperator{\cov}{cov}
\DeclareMathOperator{\corr}{corr}
\DeclareMathOperator{\var}{var}
\DeclareMathOperator{\mse}{mse}
\DeclareMathOperator{\se}{se}
\DeclareMathOperator{\row}{row}
\DeclareMathOperator{\col}{col}
\DeclareMathOperator{\range}{rng}
\DeclareMathOperator{\dimension}{dim}
\DeclareMathOperator{\bias}{bias}


% mics short cuts and symbols
\newcommand{\st}{\ensuremath{\ \mathrm{s.t.}\ }}
\newcommand{\setntn}[2]{ \{ #1 : #2 \} }
\newcommand{\cf}[1]{ \lstinline|#1| }
\newcommand{\fore}{\therefore \quad}
\newcommand{\tod}{\stackrel { d } {\to} }
\newcommand{\toprob}{\stackrel { p } {\to} }
\newcommand{\toms}{\stackrel { ms } {\to} }
\newcommand{\eqdist}{\stackrel {\textrm{ \scriptsize{d} }} {=} }
\newcommand{\iidsim}{\stackrel {\textrm{ {\sc iid }}} {\sim} }
\newcommand{\1}{\mathbbm 1}
\newcommand{\dee}{\,{\rm d}}
\newcommand{\given}{\, | \,}
\newcommand{\la}{\langle}
\newcommand{\ra}{\rangle}

\newcommand{\boldA}{\mathbf A}
\newcommand{\boldB}{\mathbf B}
\newcommand{\boldC}{\mathbf C}
\newcommand{\boldD}{\mathbf D}
\newcommand{\boldM}{\mathbf M}
\newcommand{\boldP}{\mathbf P}
\newcommand{\boldQ}{\mathbf Q}
\newcommand{\boldI}{\mathbf I}
\newcommand{\boldX}{\mathbf X}
\newcommand{\boldY}{\mathbf Y}
\newcommand{\boldZ}{\mathbf Z}

\newcommand{\bSigmaX}{ {\boldsymbol \Sigma_{\hboldbeta}} }
\newcommand{\hbSigmaX}{ \mathbf{\hat \Sigma_{\hboldbeta}} }

\newcommand{\RR}{\mathbbm R}
\newcommand{\NN}{\mathbbm N}
\newcommand{\PP}{\mathbbm P}
\newcommand{\EE}{\mathbbm E \,}
\newcommand{\XX}{\mathbbm X}
\newcommand{\ZZ}{\mathbbm Z}
\newcommand{\QQ}{\mathbbm Q}

\newcommand{\fF}{\mathcal F}
\newcommand{\dD}{\mathcal D}
\newcommand{\lL}{\mathcal L}
\newcommand{\gG}{\mathcal G}
\newcommand{\hH}{\mathcal H}
\newcommand{\nN}{\mathcal N}
\newcommand{\pP}{\mathcal P}


